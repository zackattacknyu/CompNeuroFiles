

\documentclass[conference]{IEEEtran}


\ifCLASSINFOpdf
  % \usepackage[pdftex]{graphicx}
  % declare the path(s) where your graphic files are
  % \graphicspath{{../pdf/}{../jpeg/}}
  % and their extensions so you won't have to specify these with
  % every instance of \includegraphics
  % \DeclareGraphicsExtensions{.pdf,.jpeg,.png}
\else
  % or other class option (dvipsone, dvipdf, if not using dvips). graphicx
  % will default to the driver specified in the system graphics.cfg if no
  % driver is specified.
  % \usepackage[dvips]{graphicx}
  % declare the path(s) where your graphic files are
  % \graphicspath{{../eps/}}
  % and their extensions so you won't have to specify these with
  % every instance of \includegraphics
  % \DeclareGraphicsExtensions{.eps}
\fi



% correct bad hyphenation here
\hyphenation{op-tical net-works semi-conduc-tor}


\begin{document}

\title{Planning Paths for Rats using Actor-Critic Model}


\author{Zachary DeStefano}

% make the title area
\maketitle


\begin{abstract}
For this project, I attempted to see what paths a rat would take if the Morris Water Maze contained multiple rewards of varying value. The inspiration for this came from the layout of cities and the transportation paths used between them. I wanted to see if a trained rat would take similar paths to the transportation networks that we see. In order to do this, I first trained the rat using the Actor-Critic Model. After training, I started the rat at random locations and recorded the paths it took to the reward centers. In the end, the paths are suboptimal thus rats should not be used for path planning. 
\end{abstract}

\IEEEpeerreviewmaketitle



\section{Introduction}

The Morris Water Maze is one where a rat is released into a pool of water and has to learn the location of the reward center. \\
\\
A common model for the neuronal activity in the rat involves place cells in the hippocampus that send signals depending on the current location of the rat. The Actor-Critic Model is used to control what the optimal next move should be for the rat. \\
\\
A common problem is transporation planning is that there are multiple locations that all need to reach each other so there needs to be a minimal cost network that connects them all. I attempted to see if rat movements could give more insights into this problem. \\

\section{Related Work}

\subsection{Foraging Behavior in Rats}
Foraging behavior for rats has been commonly studied.

\subsection{DA-STDP for foraging}

There were attempts to study foraging behavior using DA-STDP. In this paper, **CITE PLOS ONE PAPER**, they modeled a rat foraging for food in an environment. In the paper there were many issues getting the model to work thus it would have proven impractical for me to do. \\
\\
In this paper, **THE ROBOT**, they model a robot's movement via DA-STDP. While they accomplish a lot in the paper, due to the lack of place cells and other deficiencies with the input, it would have proven impractical for my purposes. Tuning DA-STDP would have been very difficult and deciding on good input parameters would also have been challenging. \\

\subsection{Slime Mode Interstates}

There was work done with slime molds and they were able to model the interstate highway system

\subsection{Shortest Paths in GIS}

In the field of GIS, finding the shortest path across terrain is a common problem. Additionally, finding the best route between cities is another common one. INSERT CITATIONS FOR PAPERS HERE

\section{Experiment Design}

I trained the rat\\
DETAILS ON ACTOR-CRITIC MODEL TRAINING USED\\
DETAILS ON LAYOUT AND REWARD VALUES USED\\

\section{Results}

FIGURES WITH THE PATH RESULTS AND LEARNED ACTOR-CRITIC VALUES\\
INCLUDE FIGURES SHOWING CUMULATIVE DISTANCE TO THE TARGET

\section{Conclusion}
Rats should not be used for path planning


\begin{thebibliography}{1}

\bibitem{IEEEhowto:kopka}
H.~Kopka and P.~W. Daly, \emph{A Guide to \LaTeX}, 3rd~ed.\hskip 1em plus
  0.5em minus 0.4em\relax Harlow, England: Addison-Wesley, 1999.

\end{thebibliography}




% that's all folks
\end{document}


